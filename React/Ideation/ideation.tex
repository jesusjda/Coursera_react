\documentclass[a4paper]{article}

%% Language and font encodings
\usepackage[english]{babel}
\usepackage[utf8x]{inputenc}
\usepackage[T1]{fontenc}

%% Sets page size and margins
\usepackage[a4paper,top=3cm,bottom=2cm,left=3cm,right=3cm,marginparwidth=1.75cm]{geometry}

%% Useful packages
\usepackage{amsmath}
\usepackage{amsthm}
\usepackage{amssymb}
\usepackage{graphicx}
\usepackage[colorinlistoftodos]{todonotes}
\usepackage[colorlinks=true, allcolors=blue]{hyperref}

\title{Ideation: Personal Web Page}
\author{Jes\'us J.D.A.}
%% \institution{Complutense University of Madrid}


\begin{document}
\maketitle

\begin{abstract}
Brief report describing the general idea of my personal page.
\end{abstract}


\section{Introduction}
The goal of the site is to work as an online CV and personal
information container. It will benefit me to share my results, tools,
publications and other materials I develop.

\section{Features}
The web-site will include different sections:
\begin{itemize}
\item \textbf{Main}: The main section, a brief description of my
  interests, current position, contact information and links
  to other sections.
\item \textbf{CV}: An inline CV, that has options to change the
  language and also to select different layouts depending on the
  interest of the viewer. For example, one layout will remark research
  publication, projects, teaching, etc. Other can be focused on my
  positions and studies, tech, tools..

\item \textbf{Publications}: All the information of my publications,
  with options to filter by topics, quality, date, co-authors, etc
\item \textbf{Projects} and other things: List of projects, program
  committees and other things related to research.
\item \textbf{Tools}: access to different tools with their
  web-interface.
\item \textbf{Teaching}: history of my teaching.

\item \textbf{Personal}: maybe hidden section, with log-in to access to
  a NAS server that will contain documents, photos, videos, etc.
\end{itemize}


\section{Survey}
There are a lot of personal web-sites. But any of them fits
completely. Also, one of the goals of the web-site is to show  how do
I manage with certain technologies.

\section{References}

No.

\end{document}

%%% Local Variables:
%%% mode: latex
%%% TeX-master: t
%%% End:
